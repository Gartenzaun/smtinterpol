\section{Preliminaries}

\begin{techreport}
In this section, we give an overview of what is needed to understand the
procedure we will propose in the later sections. We will briefly introduce the
logic and the theories used in this paper.
Furthermore, we define key terms like Craig interpolants and symbol sets.
\end{techreport}

\paragraph*{Logic, Theories, and SMT.}
We assume standard first-order logic. We operate within
the quantifier-free fragments of the theory of equality with uninterpreted
functions \euf and the theories of linear arithmetic over rationals \laq
and integers \laz.
The quantifier-free fragment of \laz is not closed under
interpolation.  Therefore, we augment the signature with
division by constant functions $\floorfrac{\cdot}{k}$ for all integers $k\geq 1$.

We use the standard notations $\models_T, \bot, \top$ to denote
entailment in the theory $T$, contradiction, and tautology.  In the
following, we drop the subscript $T$ as it always corresponds to the combined
theory of \euf, \laq, and \laz.

The literals in \laz are of the form $s \leq c$, where $c$ is an
integer constant and $s$ a linear combination of variables.  For \laq
we use constants $c \in \mathbb{Q}_\eps$, 
$\mathbb{Q}_\eps := \mathbb{Q} \cup \{q - \eps | q\in
\mathbb{Q}\}$ where the meaning of $s \leq q-\eps$ is $s < q$.  For
better readability we use, e.\,g., $x \leq y$ resp.\ $x > y$ to denote
$x-y\leq 0$ resp.\ $y-x \leq -\eps$.  In the integer case we use $x > y$ 
to denote $y-x \leq -1$.

Our algorithm operates on a proof of unsatisfiability generated
by an SMT solver based on DPLL$(T)$~\cite{dpllt}.
Such a proof is a resolution tree with the
$\bot$-clause at its root. The leaves of the tree are either clauses
from the input formulae\footnote{W.\,l.\,o.\,g.\ we assume input formulae are
  in conjunctive normal form.} or theory lemmas that are produced by one of
the theory solvers. The negation of a theory lemma is called a
\emph{conflict}.

The theory solvers for \euf, \laq, and \laz are working independently
and exchange (dis-)equality literals through the DPLL engine in a
Nelson-Oppen style~\cite{DBLP:journals/toplas/NelsonO79}.  Internally,
the solver for linear arithmetic uses only inequalities in
theory conflicts.  In the proof tree, the (dis-)equalities are related
to inequalities by the (valid) clauses $x=y \lor x<y \lor x>y$, and
$x\neq y \lor x\leq y$.  We call these leaves of the proof tree
\emph{theory combination clauses}.

\paragraph*{Interpolants and Symbol Sets.}

For a formula $F$, we use $\symb(F)$ to denote the set of non-theory
symbols occurring in $F$.  An interpolation problem is given by two
formulae $A$ and $B$ such that $A \land B \models \bot$.  
An interpolant of $A$ and $B$ is a formula $I$ such that 
%
(i) $A \models I$, (ii) $B \land I \models \bot$, and 
(iii) $\symb(I) \subseteq \symb(A)\cap \symb(B)$.

We call a symbol $s \in \symb(A)\cup\symb(B)$ \emph{shared}
if $s \in \symb(A) \cap \symb(B)$, \emph{$A$-local} if $s \in
\symb(A) \setminus \symb(B)$, and \emph{$B$-local} if $s \in
\symb(B) \setminus \symb(A)$.
Similarly, we call a term \emph{$A$-local} (\emph{$B$-local}) if it
contains at least one $A$-local ($B$-local) and no
$B$-local ($A$-local) symbols.
We call a term \emph{($AB$-)shared} if it contains only shared
symbols and \emph{($AB$-)mixed} if it contains $A$-local as well as $B$-local
symbols. The same terminology applies to formulae.

\paragraph*{Substitution in Formulae and Monotonicity.}
By $F[G_1]\ldots[G_n]$ we denote a formula in negation normal form with
sub-formulae $G_1,\ldots,G_n$ that occur positively in the formula.  Substituting
these sub-formulae by formula $G_1',\ldots,G_n'$ is denoted by $F[G_1']\ldots[G_n']$.  By $F(t)$ we
denote a formula with a sub-term $t$ that can appear anywhere in $F$.  The substitution of $t$ with a
term $t'$ is denoted by $F(t')$.

\begin{tacas}
The following lemma is important for the correctness proofs of our
interpolation scheme.
\end{tacas}
\begin{techreport}
The following lemma is important for the correctness proofs in the remainder
of this technical report. It also represents a concept that is important for
the understanding of the proposed procedure.
\end{techreport}

\begin{lemma}[Monotonicity]\label{lemma:monotonicity}
 Given a formula $F[G_1]\ldots[G_n]$ in negation normal form with
 sub-formulae $G_1,\ldots,G_n$ occurring only positively in
 the formula and formulae $G_1',\ldots,G_n'$, it holds that
 \[\left( \bigwedge_{i\in\{1,\ldots,n\}} (G_i \rightarrow G_i') \right)
   \rightarrow (F[G_1]\ldots[G_n] \rightarrow F[G_1']\ldots[G_n'])\]
\end{lemma}

\begin{techreport}
\begin{proof}
  We prove the claim by induction over the number of $\land$ and $\lor$
  connectives in $F[\cdot]\ldots[\cdot]$.  If $F[G_1]\ldots[G_n]$ is 
  a literal different from $G_1,\ldots,G_n$ the implication holds trivially.
  Also for the other base case $F[G_1]\ldots[G_n]\equiv G_i$ for some
  $i\in\{1,\dots,n\}$ the property holds. For the induction step
  observe that if $F_1[G_1]\ldots[G_n]\rightarrow  F_1[G'_1]\ldots[G'_n]$ and
  $F_2[G_1]\ldots[G_n]\rightarrow  F_2[G'_1]\ldots[G'_n]$, then 
  \begin{align*}
  F_1[G_1]\ldots[G_n] \land F_2[G_1]\ldots[G_n] &\rightarrow
  F_1[G'_1]\ldots[G'_n] \land F_2[G'_1]\ldots[G'_n] \text{ and }\\
  F_1[G_1]\ldots[G_n] \lor F_2[G_1]\ldots[G_n] &\rightarrow 
  F_1[G'_1]\ldots[G'_n] \lor F_2[G'_1]\ldots[G'_n].
  \end{align*}

  \vspace{-20pt}\strut\qed
\end{proof}
\end{techreport}
