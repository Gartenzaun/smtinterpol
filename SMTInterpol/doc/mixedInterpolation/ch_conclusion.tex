%----------------------------------------
\section{Conclusion and Future Work}
%----------------------------------------

We presented a novel interpolation scheme to extract Craig interpolants from
resolution proofs produced by SMT solvers without restricting the solver or
reordering the proofs.  The key ingredients of our method are virtual
purifications of troublesome mixed literals, syntactical restrictions of
partial interpolants, and specialised interpolation rules for pivoting steps
on mixed literals.

In contrast to previous work, our interpolation scheme does not need
specialised rules to deal with extended branches as commonly used in
state-of-the-art SMT solvers to solve \laz-formulae.  Furthermore, our scheme
can deal with resolution steps where a mixed literal occurs in both
antecedents, which are forbidden by other schemes~\cite{Cimatti2010,Goel2009}.

Our scheme works for resolution based proofs in the DPLL(T) context provided
there is a procedure that generates partial interpolants with our syntactic
restrictions for the theory lemmas.  We sketched these procedures for the
theory lemmas generated by either congruence closure or linear arithmetic solvers
producing Farkas proofs. 
In this paper, we limited the presentation to the combination of the theory of
uninterpreted functions, and the theory of linear arithmetic over the integers
or the reals.  Nevertheless, the scheme could be extended to support other
theories.  This requires defining the projection functions for mixed
literals in the theory, defining a pattern for 
\ifnewinterpolation\else weak and strong \fi partial
interpolants, and proving a corresponding resolution rule.

We plan to produce interpolants of different strengths using the
technique from D'Silva et al.~\cite{D'Silva2010}.  This is orthogonal to our
interpolation scheme (particularly to the 
\ifnewinterpolation partial \else weak and strong \fi
interpolants used for mixed literals).
Furthermore, we want to extend the correctness proof to show that our scheme 
works with inductive sequences of interpolants~\cite{mcmillan06lai} and tree
interpolants~\cite{HHP10}. We also plan to extend this scheme to other theories
including arrays and quantifiers.
