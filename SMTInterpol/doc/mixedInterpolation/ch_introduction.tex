%----------------------------------------
\section{Introduction}
%----------------------------------------

A Craig interpolant for a pair of formulae $A$ and $B$ whose conjunction is
unsatisfiable is a formula $I$ that follows from $A$ and whose conjunction
with $B$ is unsatisfiable.  Furthermore, $I$ only contains symbols common
to $A$ and $B$. Model checking and state space abstraction~\cite{henzinger04afp,mcmillan06lai} make
intensive use of interpolation to achieve a higher degree of automation. This
increase in automation stems from the ability to fully automatically generate
interpolants from proofs produced by modern theorem provers. 

For propositional logic, a SAT solver typically produces resolution-based
proofs that show the unsatisfiability of an error path.
Extracting Craig interpolants from such
proofs is a well understood and easy task that can be accomplished, e.\,g.,
using the algorithms of Pudl\'ak~\cite{DBLP:journals/jsyml/Pudlak97} or
McMillan~\cite{DBLP:conf/tacas/McMillan04}.  An essential property of
the proofs generated by SAT solvers is that every proof step only involves
literals that occur in the input.

This property does not hold for proofs produced by SMT solvers for formulae
in a combination of first order theories.  Such solvers produce new literals
for different reasons.  First, to combine two theory solvers, SMT solvers
exchange (dis-)equalities between the symbols common to these two theories in
a Nelson-Oppen-style theory combination.
Second, various techniques dynamically generate new literals to simplify proof
generation. Third, new literals are introduced in the context of a
branch-and-bound or branch-and-cut search for non-convex theories. The
theory of linear integer arithmetic for example is typically solved by
searching a model for the relaxation of the formula to linear rational arithmetic
and then using branch-and-cut with Gomory cuts or \emph{extended
  branches}~\cite{Dillig2011} to remove the current non-integer solution 
from the solution space of the relaxation.

The literals produced by either of these techniques only contain symbols that are already present in the input.  However, a
literal produced by one of these techniques may be \emph{mixed}\footnote{Mixed
  literals sometimes are called \emph{uncolourable}.} in the sense that it may
contain symbols occurring only in $A$ and symbols occurring only in $B$.
These
literals pose the major difficulty when extracting interpolants from proofs
produced by SMT solvers.


%% While this extraction procedure is easy and well understood
%% in the context of propositional logic, extracting interpolants from a proof
%% generated by an SMT solver is more complex. In contrast to SAT solvers, SMT
%% solvers create new literals, e.\,g., to combine multiple theories in a
%% Nelson-Oppen style or to split the solution space using cuts. These literals
%% might contain symbols local to different parts of the interpolation
%% problem. Such literals are called \emph{mixed}, or, sometimes,
%% \emph{uncolourable}. Resolution steps on mixed literals are the major
%% difficulty when extracting interpolants from proofs from SMT solvers.

In this paper, we present a scheme to compute Craig interpolants in the
presence of mixed literals.  Our interpolation scheme is based on
syntactical restrictions of \emph{partial interpolants} and specialised rules to
interpolate resolution steps on mixed literals. 
This enables us to compute interpolants in the context of a state-of-the-art SMT
solver without manipulating the proof tree or restricting the solver in any
way.
We base our presentation
on the quantifier-free fragment of the combined theory of uninterpreted
functions and linear arithmetic over the rationals or the integers.
The interpolation scheme is used in the interpolating SMT solver
SMTInterpol~\cite{toolpaper}.  
\begin{tacas}
Proofs for the theorems in this paper are
given in the technical report~\cite{atr}.
\end{tacas}

%% theory of uninterpreted functions combined with the theory of linear
%% arithmetic either over the reals or the integers. We present an interpolation
%% scheme based on syntactic restriction of \emph{partial interpolants} and
%% specialised rules to interpolate resolution steps on mixed literals. Contrary
%% to existing approaches, this scheme neither limits the inferences done by the
%% SMT solver, nor does it transform the proof tree before extracting
%% interpolants. The interpolation scheme is used in the interpolating SMT solver
%% SMTInterpol.
