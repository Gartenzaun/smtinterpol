\documentclass[a4paper,12pt]{article}
\usepackage{amsmath}
\usepackage{amsthm}
\usepackage{amssymb}
\usepackage{relsize}
\usepackage{xspace}
\usepackage{xcolor}

\newcommand\si{SMTInterpol\xspace}
\newcommand\m{\mathcal{M}}
\newcommand\ma{\ensuremath{\m^A}\xspace}
\newcommand\todo[1]{\textcolor{red}{TODO: #1}}

\title{Models for Arrays in \si}
\author{J{\"u}rgen Christ}
\date{2013/02/26}

\begin{document}
\theoremstyle{plain}
\newtheorem{corollary}{Corollary}
\theoremstyle{definition}
\newtheorem{defn}{Definition}
\newtheorem{example}{Example}[section]
\theoremstyle{remark}
\newtheorem{rem}{Remark}
\newtheorem{com}{Comment}

\maketitle

\begin{abstract}
  To fully support the theory of arrays, model production in \si has to be
  extended.  This paper describes the general ideas related to the extension
  of the models and highlights why such extensions are needed.  The document
  is intended to be an evolving documentation of the implementation ideas.
\end{abstract}

\section{Motivation}
Models produced by \si are in general partial models.  They are extended on
demand to fill in holes that are not specified by the formula.  Special logic
is used to evaluate internal function symbols like \verb|+| or \verb|=|.  This
logic has to be adapted to arrays.  The main problem here are the function
symbols \verb|=| and \verb|store|.  Since we consider the existentional theory
of arrays, the symbol \verb|=| actually compares two arrays piecewise on all
possible indices.  This comparison has to be enabled by the model.  Otherwise,
a result might look like
\verb|(= (_ as-array k!2) (store (_ as-array k!2) 4 5))|
which is not correct according to the SMTLIB standard since \verb|=| has to
return a Boolean value, i.~e., either \texttt{true} or \texttt{false}.

Finally, the \verb|store| function can be used to create new arrays.  These
new arrays have to be usable in all further model calls.  Hence, they have to
be present in the model and fully included in the congruence classes defined
in that model.

\section{Current Model Structure}
\si already supports models for uninterpreted functions and numeric data
types.  These models essentially are mappings from function symbols to so
called \emph{executable terms}.

In the simplest case, an executable term simply is a value of the
corresponding domain.  For Booleans, Integers, and Reals, the domain is
obvious.  For an uninterpreted function, \si also constructs an interpretation
of this domain.  Essentially, a domain interpretation can be seen as an
enumeration of all possible (distinct) values of that sort.  
\begin{rem}
  The current structure described in this section is tailored towards
  quantifier-free formulas.  In the context of quantifiers, we might have to
  adjust this definition to support isomorphisms between an uninterpreted type
  and an interpreted type.  For example the formula $\forall
  i,j\in\mathbb{Z}.\ f(i) \neq f(j)$ where $f\ :\ \mathbb{Z}\rightarrow U$ for
  an uninterpreted sort $U$ states an isomorphism between $\mathbb{Z}$ and
  $U$.  We could construct a model and a sort interpretation by adding a
  isomorphism function.  However, this is not the topic of this document for
  now.
\end{rem}

For function symbols that take several arguments, the model currently stores a
finite index map and a default value.  To evaluate a function application, we
first evaluate the arguments and then lookup the parameters in the index map.
If they are found, the value mapped to these parameters is the value of the
function application.  Otherwise, we resort to the default value.

Currently, the evaluation is done on terms instead of executable terms.  This
can be done since we don't have to pass information from one evaluation to the
next.  The evaluation of a term only depends on the congruence class of its
sub-terms and not on their concrete value.  This situation changes once we
want to support models for arrays.

The main requirement on sort interpretations is that they can offer a default
value for that sort.  However, there is no restriction on this element except
for the obvious requirement that it has to be part of the value allowed by the
sort interpretation.

\section{Definitions}
This section covers the main terms used in the remainder of this paper.

\begin{defn}[Base array]
  We call an array that is not composed of \verb+store+ applications
  \emph{base array}.
\end{defn}

This definition implies the following corollary.
\begin{corollary}
  Let \verb|T1| and \verb|T2| be arbitrary types.  If \verb|a| is an array
  declared as \verb|(declare-fun a () (Array T1 T2))|, then \verb|a| is a base
  array.
\end{corollary}

Similarly to the models for functions, we define the model for an array as a
finite map and a default value.  
%
\begin{defn}[Array Map]
  An \emph{array map} is a finite set of tuples $(i,v)$.
\end{defn}
%
\begin{defn}[Array Model]
  An \emph{array model} \ma consists of an \emph{array map} $M$ and a default
  value $d$.  Evaluation against these models is defined as
  \[
  \ma[i] \equiv \left\{\begin{array}{l@{\quad\mbox{if}\quad}l} v & (i,v)\in
  M\\ d & \mbox{otherwise}\end{array}\right.
  \]
\end{defn}

With this definition, we can require that a model assigns an array model to
every array term.  Hence, we can construct complete models.

\section{Constructing Models}
Model construction has to follow a strict order.  We cannot construct a model
for an array before we have constructed the models for all read and write
indices and values occurring in the input formula.  Hence, in the remainder of
this document, we assume such a model has already been constructed and we try
to construct a model for the arrays in the formula.  We will conclude this
section with some ideas on how to create a model if the strict ordering above
cannot be satisfied since an array appears as argument to a function.

\subsection{Constructing Array Models}
Model construction for arrays requires saturation of the array axioms, i.~e.,
all array axioms have to be enforced for all possible arrays, indices, and values.

Assuming the model already contains values for all non-array terms.  Then we
can construct an array model for every array term in the input as follows.
First note that we only need to define an array model for every base array.
All other array-terms in the formula can then be derived from these array
models.

To construct an array model for a base array $a$, we look at the elements in
the equivalence class of $a$ and all select-terms for $a$.  If $select(a,i)
\cong v$, we store in the array map the tuple $(rep(i), rep(v))$.  For every
store-term in the equivalence class of $a$, we add another entry to the array
map:  If $a\cong store(b,i,v)$, we add the tuple $(rep(i), rep(v))$ to the
array map.  Note that we also have to take care of nested store-terms with
distinct indices.

Finally, for every base array of sort $(Array\ T1\ T2)$ we set the default value
for $T2$ as the default value for this array.  Note that this default value
might correspond to one of the values already mapped to some index.
Especially, this index might correspond to an index from an extensionality
axiom instantiation.  Note that this is not a problem since the extensionality
axiom separates the two arrays at that position.  Let $i$ be the index used in
the instantiation of the extensionality axiom for arrays $a$ and $b$.  Then,
if $a\neq b$ holds, the instantiation requires that $select(a,i)$ is different
from $select(b,i)$.  If one of them corresponds to the default value of the
corresponding sort, the other array has to have a different value at position
$i$.  All positions not mentioned in the formula and not created by axiom
instantiations are irrelevant for model construction.

\subsection{Default Values for Array Sorts}
Creating default values for array sorts is easy as long as the value-sort has
a default value.  In this case, a default array model can be constructed as an
empty array map and the default value of the value sort, i.~e., the model maps
all possible indices to the default value of the value-sort.

Since default values are only needed to complete models of functions, usual
model completion can be used to complete interpretations.  Assume $T$ is an
uninterpreted sort, the initial interpretation of $(Array\ Int\ T)$ is empty,
and $f\ :\ Int\rightarrow (Array\ Int\ T)$ also has an empty interpretation.
Evaluating $f(73)$ simply creates a new array of sort $(Array\ Int\ T)$ as
default value for that sort.  Note that the array model for this array is
still under construction since $T$ has no interpretation by now.  Evaluating
$select(f(73),42)$ sets this term as default value for sort $T$ and updates
the array model of $f(73)$ such that it now represents the constant array of
that default value.

Compared to the current implementation, we only need to update the completion
setting despite the obvious need to implement array models.  Now, a default
value for a sort might also be created by an array selection.  Furthermore,
creation of a default value for a sort might trigger an update in several
array models that still depend on this default value.  Note that we might not
have to perform this update unless we want to print ``complete'' array
interpretations.  But in this case, we always have to generate a default
value.  Preferably, \si should not construct detailed interpretations, but
refer to the congruence roots of the corresponding arrays similar to the
procedure implemented for function symbols.

\subsection{Constructing Models with Array-Parameters}
In this section, we shortly consider model construction for functions that
take arrays as parameters.  Arrays that occur as parameters are sometimes
called \emph{foreign} as they are a source from outside the array theory that
needs additional axiom instantiations.  In this case, the most relevant axiom
is extensionality to ensure proper congruence closure of the surrounding
functions.

Assuming proper congruence closure, a model for a surrounding function symbol
requires a finite number of array models for the different congruence
classes.  The ordering restriction mentioned above dictates that function
symbols have to have a model before we can compute a model for arrays.  In
this case, a model for an array can simply be a skeleton that only marks the
root of the equivalence class of the array.  The actual values stored in this
array model are not of any interest while computing the model of the function
symbol.

\section{Need and Use of Array Models}
Since SMTLIB also allows users to evaluate whole expressions, the models
should be such that they can decide equality of arrays (including
extensionality).  Note that array models make this really easy.  Given to
arrays $a$ and $b$ with corresponding array models $\ma_a=(M_a, d)$ and
$\ma_b=(M_b, d)$.  Then, $a=b$ if and only if for all pairs $(i,v)\in M_a$
either $(i,v)\in M_b$ or $v=d\land\forall (j,w)\in M_b.\ j\neq i$ and
similarly for $M_a$ and $M_b$ swapped.  This check is polynomial in the size
of the two array models.

\todo{Open Question:  Why do people claim this is undecidable?  In my opinion,
  given a model that maps functions onto arrays, everything is fine since we
  only need to evaluate.  The general decision problem involving functions
  interpreted as arrays might be undecidable.  But I'm not sure about that.}

\end{document}
